\capitulo{7}{Conclusiones y Líneas de trabajo futuras}

Con esta sección, finaliza la documentación del proyecto, detallando los objetivos alcanzados. A continuación, se presentarán algunas posibles direcciones para investigaciones futuras que podrían ampliar y mejorar significativamente el proyecto.

\section{Conclusiones}

Al concluir el proyecto, podemos afirmar lo siguiente:

\begin{itemize}
\item Se lograron alcanzar los objetivos establecidos al inicio del proyecto, los cuales se centraban en la recopilación y visualización de datos ambientales en invernaderos de cannabis medicinal.
\item La elección de la Raspberry Pi Pico W demostró ser acertada, ya que su bajo consumo de energía, tamaño compacto y conectividad WiFi facilitaron la implementación a un costo razonable.
\item Tanto los componentes hardware como el software seleccionados son de fácil adquisición, permitiendo replicar el sistema con relativa facilidad. Esto contribuye a la accesibilidad de la tecnología en el ámbito de los invernaderos.
\item La capacidad de conectividad WiFi de la Raspberry Pi Pico W permitió la transmisión de datos y la recepción de órdenes de manera eficiente. La implementación de MQTT agregó una capa adicional de versatilidad al sistema.
\item Se logró una amplia variedad de formas de visualizar los datos ambientales, proporcionando a los usuarios una comprensión más completa y detallada de las condiciones dentro del invernadero.
\item Se identifican oportunidades para futuras mejoras, como la expansión de la capacidad de monitoreo, la integración de más sensores especializados y la implementación de alertas automáticas ante condiciones críticas.
\item Este proyecto brinda una solución económica y accesible para invernaderos de cannabis medicinal, con potencial para adaptarse a otros entornos de cultivo.
\end{itemize}

\section{Líneas de trabajo futuras}

El proyecto se ha desarrollado de manera integral, abordando de manera efectiva la funcionalidad deseada y proporcionando una solución adecuada a las necesidades planteadas. Aunque se ha logrado un producto sólido, existe el potencial para mejorar en diversos aspectos. 

\begin{itemize}
\item Integrar sensores adicionales para monitorizar parámetros específicos del entorno del invernadero, como la concentración de CO2, la calidad del suelo, o la presencia de plagas. Esto proporcionaría un monitoreo más completo y detallado.
\item Desarrollar algoritmos de análisis de datos más avanzados para identificar patrones, tendencias o anomalías en la información recopilada. Esto proporcionaría insights más profundos sobre el comportamiento del invernadero.
\item Mejorar la interfaz de usuario para proporcionar funcionalidades más avanzadas, gráficos interactivos, información estadística y herramientas de análisis. Esto facilitaría la interpretación de los datos y la toma de decisiones.
\item Integrar un sistema de riego automatizado que pueda ajustarse en tiempo real según las condiciones ambientales registradas. Esto optimizaría el consumo de agua y mejoraría la eficiencia del proceso de cultivo.
\item Extender las pruebas y adaptar el sistema para su uso en otros tipos de cultivos, permitiendo su aplicación en entornos agrícolas más amplios.
\item Reforzar la seguridad del sistema, incluyendo medidas para proteger la integridad de los datos y garantizar la privacidad de la información recopilada.
\end{itemize}

%Todo proyecto debe incluir las conclusiones que se derivan de su desarrollo. Éstas pueden ser de diferente índole, dependiendo de la tipología del proyecto, pero normalmente van a estar presentes un conjunto de conclusiones relacionadas con los resultados del proyecto y un conjunto de conclusiones técnicas. 
%Además, resulta muy útil realizar un informe crítico indicando cómo se puede mejorar el proyecto, o cómo se puede continuar trabajando en la línea del proyecto realizado. 

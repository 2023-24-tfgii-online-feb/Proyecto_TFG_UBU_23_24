\capitulo{2}{Objetivos del proyecto}
%Este apartado explica de forma precisa y concisa cuales son los objetivos que se persiguen con la realización del proyecto. Se puede distinguir entre los objetivos marcados por los requisitos del software a construir y los objetivos de carácter técnico que plantea a la hora de llevar a la práctica el proyecto.
\section{Objetivos del Software}
\begin{itemize}
\item \textbf{Desarrollo del Sistema de Adquisición de Datos:}
Implementar un sistema eficiente de adquisición de datos que pueda recopilar información precisa proveniente de los sensores (DHT22, BH1750, sensor de humedad de suelo) instalados en el invernadero.
\item \textbf{Diseño de la Interfaz de Usuario:}
Desarrollar una interfaz de usuario intuitiva y fácil de usar, que permita a los usuarios visualizar en tiempo real los datos recopilados y tomar decisiones informadas sobre el control del entorno del invernadero.
\item \textbf{Implementación de Protocolos de Comunicación:}
Establecer protocolos de comunicación eficientes para la transmisión de datos entre los sensores y la Raspberry Pi Pico W, así como para la conexión con otros dispositivos o sistemas externos si es necesario.
\end{itemize}
\pagebreak

\section{Objetivos Técnicos}
\begin{itemize}
\item \textbf{Integración de Hardware:} 
Seleccionar, configurar e integrar de manera óptima el hardware necesario (Raspberry Pi Pico W, pantalla OLED 128x64, sensores) para garantizar la estabilidad y la eficacia del sistema.

\item \textbf{Eficiencia Energética:}
Implementar estrategias de programación y configuración de hardware que optimicen el consumo de energía, asegurando una operación sostenible del sistema en entornos con recursos limitados. Precisamente la Raspberry Pi Pico W es conocida por su bajo consumo energético.

\item \textbf{Validación en Entorno Real:}
Realizar pruebas exhaustivas del sistema en un entorno de cultivo de cannabis medicinal real, evaluando su rendimiento, confiabilidad y adaptabilidad a condiciones variables.

\item \textbf{Consideraciones Económicas:}
Evaluar y optimizar los costos asociados al desarrollo e implementación del sistema, garantizando la viabilidad económica para agricultores de diferentes escalas y recursos.

\item \textbf{Documentación Completa:}
Generar una documentación detallada que abarque el diseño, la implementación y la configuración del sistema, facilitando la comprensión y la replicación por parte de otros interesados.
Estos objetivos, tanto del software como técnicos, se plantean con el propósito de cumplir con los requisitos del proyecto y garantizar la efectividad y utilidad del sistema de monitorización propuesto para invernaderos de cannabis medicinal.
\end{itemize}

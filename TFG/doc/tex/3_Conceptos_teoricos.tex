\capitulo{3}{Conceptos teóricos}

%En aquellos proyectos que necesiten para su comprensión y desarrollo de unos conceptos teóricos de una determinada materia o de un determinado dominio de conocimiento, debe existir un apartado que sintetice dichos conceptos.

%Algunos conceptos teóricos de \LaTeX{} \footnote{Créditos a los proyectos de Álvaro López Cantero: Configurador de Presupuestos y Roberto Izquierdo Amo: PLQuiz}.

%\section{Referencias}
%Las referencias se incluyen en el texto usando cite~\cite{wiki:latex}. Para citar webs, artículos o libros~\cite{koza92}, si se desean citar más de uno en el mismo lugar~\cite{bortolot2005, koza92}.

%\imagen{escudoInfor}{Autómata para una expresión vacía}{.5} %imagenes
	
%\section{Tablas}
%Igualmente se pueden usar los comandos específicos de \LaTeX o bien usar alguno de los comandos de la plantilla.
%\tablaSmall{Herramientas y tecnologías utilizadas en cada parte del proyecto}{l c c c c}{herramientasportipodeuso}
%{ \multicolumn{1}{l}{Herramientas} & App AngularJS & API REST & BD & Memoria \\}{ 
%HTML5 & X & & &\\
%CSS3 & X & & &\\
%BOOTSTRAP & X & & &\\
%JavaScript & X & & &\\
%AngularJS & X & & &\\
%Bower & X & & &\\
%PHP & & X & &\\
%Karma + Jasmine & X & & &\\
%Slim framework & & X & &\\
%Idiorm & & X & &\\
%Composer & & X & &\\
%JSON & X & X & &\\
%PhpStorm & X & X & &\\
%MySQL & & & X &\\
%PhpMyAdmin & & & X &\\
%Git + BitBucket & X & X & X & X\\
%Mik\TeX{} & & & & X\\
%\TeX{}Maker & & & & X\\
%Astah & & & & X\\
%Balsamiq Mockups & X & & &\\
%VersionOne & X & X & X & X\\
%} 
\section{Domótica}\label{concepto:Domótica}
La domótica se refiere a la aplicación de tecnologías de la información y la comunicación (TIC) para automatizar y gestionar de manera inteligente diferentes aspectos de una vivienda o edificio. El objetivo principal de la domótica es mejorar la calidad de vida, proporcionando mayor comodidad, seguridad, eficiencia energética y accesibilidad a sus habitantes.

Los sistemas domóticos suelen integrar dispositivos electrónicos, sensores, actuadores y redes de comunicación para permitir la supervisión y control remoto de las funciones del hogar. Estas funciones pueden incluir la gestión de la iluminación, calefacción, ventilación, sistemas de seguridad, electrodomésticos, entre otros. A través de la interconexión de estos elementos, los usuarios pueden controlar y programar diversas tareas de forma centralizada, a menudo a través de interfaces intuitivas como aplicaciones móviles o paneles de control.

La domótica no solo busca la automatización de tareas cotidianas, sino también la adaptación del entorno a las necesidades y preferencias individuales. Además, puede contribuir significativamente a la eficiencia energética al permitir la optimización del consumo de recursos en el hogar. En resumen, la domótica se centra en crear ambientes inteligentes que respondan de manera proactiva y eficiente a las demandas y rutinas de los usuarios.

\section{Internet de las cosas (IoT)}
Se refiere a la interconexión de dispositivos físicos a través de internet, permitiéndoles recopilar y compartir datos.


\section{Raspberry Pi Pico W}
La Raspberry Pi Pico W~\cite{misc:RPiPicoW} es una placa de desarrollo de bajo costo que combina versatilidad y conectividad inalámbrica. Se elige como unidad central para el sistema, actuando como el cerebro que controla y coordina la recopilación y transmisión de datos.

\section{MicroPython}
MicroPython~\cite{wiki:micropython} es una implementación del lenguaje de programación Python diseñada para sistemas embebidos y microcontroladores como la Raspberry Pi Pico W. Permite una programación sencilla y eficiente de la unidad central, facilitando el desarrollo de aplicaciones en entornos limitados.

\section{GPIO}\label{concepto:GPIO}
GPIO~\cite{misc:gpio}, por sus siglas en inglés, General Purpose Input/Output, se refiere a los pines de propósito general presentes en microcontroladores y sistemas embebidos. Estos pines son versátiles y pueden configurarse tanto como entradas como salidas, permitiendo la interacción del microcontrolador con el entorno externo.
\begin{itemize}
	\item \textbf{Entradas (Input):} En modo de entrada, un pin GPIO puede recibir señales eléctricas del exterior, como las provenientes de sensores o interruptores. Estos pines detectan y leen el estado de la señal, que puede ser alto (1) o bajo (0), dependiendo de la presencia de voltaje.

	\item \textbf{Salidas (Output):} En modo de salida, un pin GPIO puede enviar señales eléctricas al exterior, como las necesarias para controlar actuadores, LEDs u otros dispositivos. El microcontrolador puede controlar el estado del pin, estableciendo un voltaje alto o bajo según sea necesario.
\end{itemize}
La flexibilidad de los pines GPIO permite a los desarrolladores adaptar el comportamiento del microcontrolador a una amplia variedad de aplicaciones. La manipulación directa de estos pines a través de código facilita la interconexión con componentes externos y la implementación de funciones específicas en proyectos de hardware. En el contexto de la Raspberry Pi Pico W~\cite{misc:RPiPicoW} y otros dispositivos similares, la gestión de los pines GPIO es esencial para controlar y coordinar la interacción con sensores, actuadores y otros periféricos.

\section{Sensores (DHT22, BH1750, Sensor de Humedad de Suelo)}
Los sensores seleccionados desempeñan roles cruciales en la monitorización del entorno del invernadero. El DHT22~\cite{manual:DHT22} mide la temperatura y humedad ambiente, el BH1750~\cite{manual:BH1750} evalúa la intensidad lumínica, y el sensor de humedad de suelo~\cite{wiki:SensorHumedadSuelo} monitorea las condiciones de la tierra.

\section{Interfaz de Usuario}
La interfaz de usuario proporciona una plataforma visual para la interpretación de datos recopilados. Su diseño debe ser intuitivo, permitiendo acceder y comprender fácilmente la información relevante sobre el estado del invernadero.

\section{Protocolos de Comunicación}
Los protocolos de comunicación son conjuntos de reglas y estándares que especifican cómo los dispositivos deben intercambiar información entre sí. Estos protocolos definen el formato, la secuencia y el significado de los mensajes que se envían y reciben, facilitando la comunicación efectiva entre sistemas o dispositivos electrónicos. Los protocolos son esenciales para garantizar la interoperabilidad y la correcta transmisión de datos en entornos tecnológicos diversos.

Existen diferentes tipos de protocolos de comunicación, cada uno diseñado para propósitos específicos y adaptado a diferentes situaciones. Algunos ejemplos comunes incluyen:
\begin{itemize}
\item \textbf{Protocolos de Red}:
Estos protocolos se utilizan para la transmisión de datos a través de redes de computadoras. Ejemplos incluyen el Protocolo de Internet (IP) para direccionamiento y enrutamiento, y el Protocolo de Control de Transmisión (TCP) para la gestión de conexiones fiables.

\item \textbf{Protocolos de Comunicación Inalámbrica:}
	Para la comunicación sin cables, existen protocolos específicos como el estándar IEEE 802.11 (Wi-Fi)~\cite{manual:IEEE802.11} para redes inalámbricas y el Bluetooth para la conexión de dispositivos cercanos.

\item \textbf{Protocolos de Comunicación Serie:}
	En entornos donde se requiere una comunicación punto a punto, se utilizan protocolos serie como UART (Universal Asynchronous Receiver-Transmitter)~\cite{manual:UART} y SPI (Serial Peripheral Interface)~\cite{manual:SPI-I2C}.

\item \textbf{Protocolos de Comunicación de Periféricos:}
Para la interconexión de periféricos y componentes electrónicos, se utilizan protocolos como I2C (Inter-Integrated Circuit) y SPI para la transmisión de datos entre microcontroladores, sensores y otros dispositivos.

\item \textbf{Protocolos de Aplicación:}
	Estos protocolos se centran en la comunicación a nivel de aplicación y pueden incluir estándares como HTTP (Hypertext Transfer Protocol)~\cite{manual:HTTP} para la transmisión de datos en la web y MQTT (Message Queuing Telemetry Transport)~\cite{manual:MQTT} para la comunicación en el Internet de las cosas (IoT).
\end{itemize}

En cuanto a normativas, algunas organizaciones como el IEEE (Instituto de Ingenieros Eléctricos y Electrónicos)~\cite{misc:IEEE} establecen estándares para varios protocolos de comunicación, garantizando la consistencia y la interoperabilidad en la industria. Por ejemplo, el IEEE 802.3 ~\cite{misc:IEEE802_3} define estándares para Ethernet, mientras que el IEEE 802.15 ~\cite{misc:IEEE802_15} se centra en protocolos para redes inalámbricas de área personal (WPAN), como Bluetooth. Estas normativas son esenciales para la creación de sistemas y dispositivos que puedan comunicarse de manera efectiva y sin problemas.

\section{WiFi}\label{concepto:WIFI}
Wi-Fi, derivado de "Wireless Fidelity," es una tecnología de comunicación inalámbrica que permite la conexión de dispositivos electrónicos a una red local o a Internet sin necesidad de cables físicos. Esta tecnología utiliza ondas de radio para la transmisión de datos entre dispositivos compatibles, siguiendo los estándares establecidos por el Instituto de Ingenieros Eléctricos y Electrónicos (IEEE), específicamente dentro de la familia de normas 802.11.

La norma \underline{IEEE 802.11}~\cite{manual:IEEE802.11} abarca diversas versiones, cada una con mejoras y características específicas. Por ejemplo, las variantes más comunes incluyen 802.11a, 802.11b, 802.11g, 802.11n, 802.11ac, y 802.11ax. Cada una de estas normas define aspectos técnicos como la velocidad de transmisión, el rango de frecuencias utilizadas, y las características de seguridad.

La conexión a una red Wi-Fi implica que los dispositivos inalámbricos cumplen con estas normas para garantizar una comunicación eficiente y segura. Los routers Wi-Fi, que actúan como puntos de acceso, son fundamentales para establecer y gestionar estas conexiones inalámbricas, permitiendo que múltiples dispositivos se conecten y compartan recursos en un área determinada.

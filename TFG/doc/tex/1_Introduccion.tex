\capitulo{1}{Introducción}

El panorama en constante cambio de la industria del cannabis medicinal ha generado una creciente necesidad de optimizar los métodos de cultivo para garantizar la consistencia y calidad de los productos. En este contexto, el Internet de las cosas (IoT) emerge como una herramienta clave para transformar la gestión de invernaderos, permitiendo una monitorización en tiempo real y decisiones informadas por parte de los agricultores.

Este trabajo se centra en el diseño de un sistema asequible de monitorización basado en IoT para invernaderos de cannabis medicinal. La elección de hardware, liderada por la Raspberry Pi Pico W como unidad central, se fundamenta en la capacidad de esta plataforma para equilibrar eficiencia y costos. Sensores especializados, como el DHT22 para temperatura y humedad, el BH1750 para intensidad lumínica, y un sensor de humedad de suelo, complementan la infraestructura del sistema.

Más allá de la mejora en la eficiencia del cultivo de cannabis medicinal, este proyecto busca situarse en la vanguardia de la agricultura inteligente y sostenible. La combinación de IoT con prácticas agrícolas avanzadas tiene el potencial de transformar radicalmente la forma en que se gestionan los cultivos, promoviendo la sostenibilidad y ofreciendo una solución práctica para agricultores de diversos niveles de recursos.

En resumen, este trabajo se adentra en el diseño de un sistema completo y accesible para la monitorización de invernaderos, utilizando la innovación tecnológica como catalizador para mejorar la calidad de los cultivos y contribuir al avance de la agricultura moderna.

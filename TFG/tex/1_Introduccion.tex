\capitulo{1}{Introducción}

El panorama en constante cambio de la industria del cannabis medicinal ha generado una creciente necesidad de optimizar los métodos de cultivo para garantizar la consistencia y calidad de los productos. En este contexto, el Internet de las cosas (IoT) emerge como una herramienta clave para transformar la gestión de invernaderos, permitiendo una monitorización en tiempo real y decisiones informadas por parte de los agricultores.

Este trabajo se centra en el diseño de un sistema asequible de monitorización basado en IoT para invernaderos de cannabis medicinal. La elección de hardware, liderada por la Raspberry Pi Pico W como unidad central, se fundamenta en la capacidad de esta plataforma para equilibrar eficiencia y costos. Sensores especializados, como el DHT22 para temperatura y humedad, el BH1750 para intensidad lumínica, y un sensor de humedad de suelo, complementan la infraestructura del hardware.

A través de un servidor LAMP, se centraliza la recopilación de datos en una base de datos para consultas históricas, mientras que MQTT facilita la obtención de datos en tiempo real. La información recabada puede ser visualizada de diversas maneras. Una pantalla OLED incorporada in situ en el invernadero ofrece una visualización local de los datos. Asimismo, un bot de Telegram permite interactuar mediante comandos o recibir alertas programadas, posibilitando una gestión remota eficiente.

Además, se dispone de un dashboard web que posibilita la visualización de datos tanto en tiempo real como en su historial. Complementando estas opciones, una aplicación de escritorio para Windows ofrece flexibilidad al permitir la visualización de datos en tiempo real y acceso al historial, además de proporcionar la capacidad de enviar comandos a la Raspberry Pi Pico W para activar mecanismos vinculados a las variables medidas. Esta diversidad de interfaces asegura que los usuarios tengan opciones versátiles para interactuar con el sistema de monitorización, adaptándose a sus necesidades específicas y facilitando la gestión efectiva del invernadero.

También se identificó la necesidad de utilizar Python para llevar a cabo la limpieza de datos y generar un resumen estadístico básico. A medida que la cantidad de datos aumente, se vislumbra la oportunidad de explorar posibilidades más avanzadas y significativas en el análisis de la información recopilada. Este enfoque proporciona una base sólida para futuras investigaciones y mejoras, permitiendo una evolución progresiva en la interpretación y aprovechamiento de los datos generados por el sistema.

Más allá de la mejora en la eficiencia del cultivo de cannabis medicinal, este proyecto busca situarse en la vanguardia de la agricultura inteligente y sostenible. La combinación de IoT con prácticas agrícolas avanzadas tiene el potencial de transformar radicalmente la forma en que se gestionan los cultivos, promoviendo la sostenibilidad y ofreciendo una solución práctica para agricultores de diversos niveles de recursos.

En resumen, este trabajo se adentra en el diseño de un sistema completo y accesible para la monitorización de invernaderos, utilizando la innovación tecnológica como catalizador para mejorar la calidad de los cultivos y contribuir al avance de la agricultura moderna.
